%Calculus Homework
\documentclass[a4paper, 12pt]{article}

%================================================================================
%Package
	\usepackage{amsmath, amsthm, amssymb, latexsym, mathtools, mathrsfs, physics, amsfonts}
	\usepackage{dsfont, txfonts, soul, stackrel, tikz-cd, graphicx, titlesec, etoolbox}
	\DeclareGraphicsExtensions{.pdf,.png,.jpg}
	\usepackage{fancyhdr}
	\usepackage[shortlabels]{enumitem}
	\usepackage[pdfmenubar=true, pdfborder	={0 0 0 [3 3]}]{hyperref}
	\usepackage{kotex}

%================================================================================
\usepackage{verbatim}
\usepackage{physics}
\usepackage{makebox}
\usepackage{pst-node}

%================================================================================
%Layout
	%Page layout
	\addtolength{\hoffset}{-50pt}
	\addtolength{\headheight}{+10pt}
	\addtolength{\textwidth}{+75pt}
	\addtolength{\voffset}{-50pt}
	\addtolength{\textheight}{+75pt}
	\newcommand{\Space}{1em}
	\newcommand{\Vspace}{\vspace{\Space}}
	\newcommand{\ran}{\textrm{ran }}
	\setenumerate{listparindent=\parindent}

%================================================================================
%Statement
	\newtheoremstyle{Mytheorem}%
	{1em}{1em}%
	{\slshape}{}%
	{\bfseries}{.}%
	{ }{}

	\newtheoremstyle{Mydefinition}%
	{1em}{1em}%
	{}{}%
	{\bfseries}{.}%
	{ }{}

	\theoremstyle{Mydefinition}
	\newtheorem{statement}{Statement}
	\newtheorem{definition}[statement]{Definition}
	\newtheorem{definitions}[statement]{Definitions}
	\newtheorem{remark}[statement]{Remark}
	\newtheorem{remarks}[statement]{Remarks}
	\newtheorem{example}[statement]{Example}
	\newtheorem{examples}[statement]{Examples}
	\newtheorem{question}[statement]{Question}
	\newtheorem{questions}[statement]{Questions}
	\newtheorem{problem}[statement]{Problem}
	\newtheorem{exercise}{Exercise}[section]
	\newtheorem*{comment*}{Comment}
	%\newtheorem{exercise}{Exercise}[subsection]

	\theoremstyle{Mytheorem}
	\newtheorem{theorem}[statement]{Theorem}
	\newtheorem{corollary}[statement]{Corollary}
	\newtheorem{corollaries}[statement]{Corollaries}
	\newtheorem{proposition}[statement]{Proposition}
	\newtheorem{lemma}[statement]{Lemma}
	\newtheorem{claim}{Claim}
	\newtheorem{claimproof}{Proof of claim}[claim]
	\newenvironment{myproof1}[1][\proofname]{%
  \proof[\textit Proof of problem #1]%
}{\endproof}

%================================================================================
%Header & footer
	\fancypagestyle{myfency}{%Plain
	\fancyhf{}
	\fancyhead[L]{}
	\fancyhead[C]{}
	\fancyhead[R]{}
	\fancyfoot[L]{}
	\fancyfoot[C]{}
	\fancyfoot[R]{\thepage}
	\renewcommand{\headrulewidth}{0.4pt}
	\renewcommand{\footrulewidth}{0pt}}

	\fancypagestyle{myfirstpage}{%Firstpage
	\fancyhf{}
	\fancyhead[L]{}
	\fancyhead[C]{}
	\fancyhead[R]{}
	\fancyfoot[L]{}
	\fancyfoot[C]{}
	\fancyfoot[R]{\thepage}
	\renewcommand{\headrulewidth}{0pt}
	\renewcommand{\footrulewidth}{0pt}}

	\pagestyle{myfency}

%================================================================================

%***************************
%*** Additional Command ****
%***************************

\DeclareMathOperator{\cl}{cl}
\DeclareMathOperator{\sgn}{sgn}
\DeclareMathOperator{\co}{co}
\DeclareMathOperator{\ball}{ball}
\DeclareMathOperator{\wk}{wk}
\DeclareMathOperator{\spn}{span}
\DeclareMathOperator{\Ind}{Ind}
\DeclareMathOperator{\Hom}{Hom}
\DeclarePairedDelimiter{\ceil}{\lceil}{\rceil}
\DeclarePairedDelimiter\floor{\lfloor}{\rfloor}
\newcommand{\quotZ}[1]{\ensuremath{\mathbb{Z}/p^{#1}\mathbb{Z}}}
%================================================================================
%Document
\begin{document}
\thispagestyle{myfirstpage}
\begin{center}
	\Large{temp}
\end{center}
박성빈, 수학과, 20202120

\noindent \textbf{1}(\textbf{S} 6.7)
Since the eigenvalues of $\rho(s)$ have absolute value $1$,
\begin{equation}
    \abs{\chi(s)} = \abs{\sum_{i=1}^n \lambda_i}\leq \sum_{i=1}^n\abs{\lambda_i} = n.
\end{equation}
The equality holds if any only if $\lambda_i = \sigma_i \lambda$ for some $\lambda\in \mathbb{C}$ with $\abs{\lambda} = 1$ and $\sigma_i = \pm 1$ with $\sigma_i\sigma_j>0$ for all $i,j$. It shows that $\rho(s)$ is homothety if and only if $\abs{\chi(s)}\leq n$. If $\chi(s) = 1$, then $\lambda_i=1$ and $\rho(s)$ is a homothety, which means that $\rho(s) = 1$.\\

\noindent \textbf{2}(\textbf{S} 6.8)
I'll use Galois theory to solve this problem. For each $\lambda_i$, set $[\mathbb{Q}(\lambda_i):\mathbb{Q}] = m_i$. Consider the multiple of irreducible polynomials having distinct root $\lambda_i$, we get a separable polynomial $p(x)$, and let $K$ be the splitting field of $p(x)$. Now, we get the Galois extension $K/\mathbb{Q}$. Also, the $\mathrm{Gal}(K/\mathbb{Q})$ is a subgroup of $\prod_{i}\mathrm{Gal}(\mathbb{Q}(\lambda_i)/\mathbb{Q})$.

Now, let $L$ be the splitting field of the irreducible monic polynomial having root $\xi=n^{-1}\sum_{i=1}^n \lambda_i$. (Since $\xi$ is an algebraic integer, such monic polynomial should exist.) Since $\xi\in K$, $L$ is a subfield of $K$, so a subgroup of $\mathrm{Gal}(K/\mathbb{Q})$. It means that the other roots of the irreducible polynomial is again a sum of roots of unity. Let $\{\sigma_i\} = \mathrm{Gal}(L/\mathbb{Q})$, then $\eta = \prod_{i} \sigma_i(\xi)\in\mathbb{Q}$. Furthermore, $\eta$ is an algebraic integer since it is the multiple of algebraic integers, which is the sum of root of unity. Finally, each $\abs{\sigma_i(\xi)}\leq 1$, so $\abs{\eta}\leq 1$. It shows $\eta = \pm 1$ or $0$. If $\abs{\eta} = 1$, $\abs{\xi} = 1$ and we get $a = \lambda_i$. If $\eta = 0$, then there exists some $\sigma_i$ such that $\sigma_i(\xi)= 0$. Then, $\sigma_{i}^{-1}\left(\sigma_i(\xi)\right) = \xi = 0$.\\

\noindent \textbf{3}(\textbf{S} 6.9)
Let $c(s)$ be the number of elements conjugate with $s$. It is integer and $c(s) = c(s')$ if $s$ and $s'$ are in a same conjugate class. By cor 16.1, we know that $\sum_{s\in G}\frac{c(s)}{n}\chi(s)$ is an algebraic integer.

Since $\chi(s)$ and $\frac{c(s)}{n}\chi(s)$ are algebraic integer, $\mathbb{Z}\left[\chi(s), \frac{c(s)}{n}\chi(s)\right]$ which has ring structure is finitely generated as $\mathbb{Z}$ module. Since $\frac{1}{n}\chi(s)\in \mathbb{Z}\left[\chi(s), \frac{c(s)}{n}\chi(s)\right]$, $\mathbb{Z}\left[\frac{1}{n}\chi(s)\right]$ is contained $\mathbb{Z}\left[\chi(s), \frac{c(s)}{n}\chi(s)\right]$ and $\frac{1}{n}\chi(s)$ is an algebraic integer. By the previous problems, $\rho(s)$ is a homothety.\\

\noindent \textbf{4}(\textbf{S} 6.10)
Let $c(s) = p^n$ for some $n\in\mathbb{Z}_{\geq 0}$. For a trivial character $1_G$ and an irreducible character $\chi$ not equal to trivial character (if exists),
\begin{equation}
    \sum_{i=1}^h n_i\chi_i(s) = 1 + \sum_{\chi\neq 1}\chi(1)\chi(s) = g\delta_{s}.
\end{equation}
Since $s\neq 1$, the RHS is zero. If there is no character satisfying $\chi(s)\neq 0$ and $\chi(1)\not\equiv 0\mod p$,
\begin{equation}
    1 + \sum_{\chi\neq 1}\chi(1)\chi(s) = 1+p\sum_{\chi\neq 1}(\chi(1)/p)\chi(s) = 0.
\end{equation}
Since $(\chi(1)/p)\in\mathbb{Z}$ and $\chi(s)$ is the sum of root of unities, so algebraic integers; therefore, $\sum_{\chi\neq 1}(\chi(1)/p)\chi(s)$ is an algebraic integer. It shows that $1/p$ is an algebraic integer, which is not true. Therefore, there should exist such irreducible character. Since $c(s)$ and $n$ are relatively prime and $\chi(s)\neq 0$, $\rho(s)$ is an homothety by the ex. 6.9. Assume the $\rho$ is not a unit representation, so the kernel is not $G$, then $\rho$ is well-defined on $G/N$, and $\rho(s)$ belongs to the center of $G/N$...\\

\noindent \textbf{5}(\textbf{S} 7.1)
\begin{enumerate}
    \item[(a)]Let $f$ (resp. $g$) be a representation of $H$ (resp. $G$) generating the character $\psi$ (resp. $\varphi$). $\psi$ and $\Res_\alpha \varphi$, are characters of $H$ since $g\circ \alpha$ generates the character $\Res_\alpha \varphi$. Therefore, we get
\begin{equation}
    \begin{split}
        \langle \psi, \Res_\alpha \varphi\rangle_H &= \dim_\mathbb{C} \Hom^H(W,\Res_\alpha E)\\
        \langle \Ind_\alpha\psi,  \varphi\rangle_G &= \dim_\mathbb{C} \Hom^G(\mathbb{C}[G]\otimes_{\mathbb{C}[H]}W, E).
    \end{split}
\end{equation}
I'll construct a isomorphism $\Psi$ from $\Hom^H(W,\Res_\alpha E)$ to $\Hom^G(\mathbb{C}[G]\otimes_{\mathbb{C}[H]}W, E)$. For $h\in \Hom^H(W,\Res_\alpha E)$, set $\Psi(h)(g\otimes w) = g\cdot h(w)$. I'll first show that this is well-defined: for $h':\mathbb{C}[G]\times W\rightarrow E$ by setting $h':(g, w)\mapsto g\cdot h(w)$, it is $\mathbb{C}$-bilinear, in fact, it is $\mathbb{C}[H]$-bilinear since for $s\in H$,
\begin{equation}
    h'(g\alpha(s), w) = (g\alpha(s))\cdot h(w) = g\cdot(\alpha(s)\cdot h(w)) = g\cdot h(sw) = h'(g, sw).
\end{equation}
Therefore, $h'$ can be extend to $\mathbb{C}[G]\otimes_{\mathbb{C}[H]}W$ and by the uniqueness, it is $\Psi(h)$. By giving $\mathbb{C}[G]$ action on $\Psi(h)$ by $g_1\cdot \left(\Psi(h)(g_2\otimes w)\right) = \Psi(h)(g_1g_2\otimes w)$, it is in $\Hom^G(\mathbb{C}[G]\otimes_{\mathbb{C}[H]}W, E)$.

Conversely, let $h'\in \Hom^G(\mathbb{C}[G]\otimes_{\mathbb{C}[H]}W, E)$, then construct $h$ by $h(w) = h'(1\otimes w)$ with $H$ action given by $h(sw) = h'(1\otimes sw) = h'(\alpha(s)\otimes w) = \alpha(s)h'(1\otimes w) = sh(w)$ since $h(w)\in \Res_\alpha E$. Finally, $\Psi(h)(g\otimes w) = g\cdot h(w) = g\cdot \left(h'(1\otimes w)\right) = h'(g\otimes w)$. It shows $\Psi$ is an isomorphsim and we get the result.

    \item[(b)] Let $W^0$ be the subspace of $W$ stable under $N$. It exists since $0\in W^0$. Consider a map $\psi:\mathbb{C}[G]\times W\rightarrow \mathbb{C}[G]\otimes_{\mathbb{C}[G]} W^0$ by $\psi:(gN,w)\rightarrow gN\otimes \left(\sum_{n\in N}nw\right)$. It is $\mathbb{C}[H]$ bilinaer: for $h=g'n'$ for some $h\in H$, $g'\in H/N$, and $n'\in N$,
    \begin{equation}
    \begin{split}
        \psi((gh, w)) &= ghN\otimes \left(\sum_{n\in N}nw\right) = gg'N\otimes \left(\sum_{n\in N}nw\right)\\
        \psi((g, hw)) &= gN\otimes \left(\sum_{n\in N}nhw\right) = gN\otimes (g'N)\left(\sum_{n\in N}nw\right) = gg'N\otimes \left(\sum_{n\in N}nw\right).
    \end{split}
    \end{equation}
    Therefore, by the universal property, there exists a unique homomorphism $\Psi:\mathbb{C}[G]\otimes_{\mathbb{C}[H]}W\rightarrow\mathbb{C}[G]\otimes_{\mathbb{C}[G]}W^0$. If I choose $w\in W^0$, $\psi(gN,w) = gN\otimes w$, so $\Psi$ is surjective. Also, if $\sum_i c_i g_iN\otimes w_i\mapsto 0$, it means
    \begin{equation}
        \sum_i c_i\otimes g_i\left(\sum_{n\in N}nw_i\right) = \sum_i c_i\otimes \left(\sum_{n\in N}ng_iw_i\right) = 0
    \end{equation}
    and
    \begin{equation}
        \sum_i c_i g_iN\otimes w_i = \sum_i c_i g_iN g_i^{-1}\otimes g_i w_i = \sum_i c_i N\otimes g_i w_i = \sum_i c_i N\otimes ng_i w_i,
    \end{equation}
    for all $n\in N$, so ... bijective and isomorphism.
    
\end{enumerate}




\begin{equation}
    \langle \psi, \Res_\alpha \varphi\rangle_H = \frac{1}{h}\sum_{s\in H}\psi(s^{-1})\left(\varphi\circ\alpha(s)\right)
\end{equation}
\begin{equation}
\begin{split}
    \langle \Ind_\alpha\psi, \varphi\rangle_G &= \frac{1}{gh}\sum_{s\in G}\sum_{\substack{t\in G}}\psi(\alpha^{-1}(ts^{-1}t^{-1}))\varphi(s) = \frac{1}{gh}\sum_{s\in G}\sum_{\substack{t\in G}}\psi(\alpha^{-1}(s^{-1}))\varphi(t^{-1}st)\\
    &=\frac{1}{gh}\sum_{s\in \alpha(H)}\sum_{\substack{t\in G}}\psi(\alpha^{-1}(s^{-1}))\varphi(s) = \frac{1}{gh}\sum_{s\in \alpha(H)}\sum_{\substack{t\in G}}\psi(\alpha^{-1}(s^{-1}))\varphi(s) = \frac{1}{h}\sum_{s\in H}\sum_{\substack{t\in G}}\psi(s^{-1})\varphi(\alpha(s))
\end{split}
\end{equation}

\begin{equation}
    \dim\Hom^H(W,E) = \dim\Hom^G(\mathbb{C}[G]\otimes_{\mathbb{C}[H]}W,E)
\end{equation}


\noindent \textbf{1}(\textbf{S} 7.2)
Using the definition of induced class function,
\begin{equation}
    \Ind_H^G(1_H)(s) = h^{-1}\sum_{\substack{t\in G\\ t^{-1}st\in H}}1_H(t^{-1}st).
\end{equation}
Now, let's compute the number of left cosets in $G/H$ fixed by action $s$. It is the number of $t\in G$ divided by $h$ such that $t^{-1}st\in H$...

Now, we know $\Ind_H^G(1_H) = \chi$ and $\Res_H 1_G = 1_H$. Using the Frobenius reciprocity formula,
\begin{equation}
    1 = \langle 1_H, \Res_H 1_G\rangle = \langle \chi, 1_G\rangle.
\end{equation}
It means that the representation generating the character $\chi$ has one unit irreducible representation. Decomposing the representation into irreducible ones and we get $\chi = 1\oplus \theta$ for some (not necessarily irreducible) representation $\theta$ and the character of $\theta$ is $\chi-1$. (In fact, it may be 0: if $H=G$, then $\chi = 1_G$, so $\theta$ is $0$ representation. If $[G:H]\geq 2$, then it has dimension bigger than $1$, so $\theta$ is not $0$ representaion.)

From the exercise 2.6, $\chi-1$ is irreducible if and only if $G$ acts on $G/H$ doubly transitively, or $\langle \chi, \chi\rangle = 2$: note that $\chi$ is real-valued... Definitely, $H$ is not a normal subgroup of $G$ if it is doubly transitive on $G/H$.

\noindent \textbf{2}-\textbf{8}(\textbf{S} 7.3)
\begin{enumerate}
    \item[(a)] Since $H\cap tHt^{-1} = 1$ for all $t\not\in H$, for a represetation $\{r_i\}\subset G$ of $G/H$ with $r_1 = 1$, $H\cap r_iHr_i^{-1} = 1$ for $i\geq 2$. Also, for $i\neq j$ with $i,j\geq 2$, $H\cap r_1^{-1}r_2H(r_1^{-1}r_2)^{-1} = 1$, so $r_1Hr_1^{-1}\cap r_2Hr_2^{-1} = 1$. Finally, for any $t = r_ih\in G$, $tHt^{-1} = r_ihHh^{-1}r_i^{-1} = r_iHr_i^{-1}$. It shows that
    \begin{equation}
        N = G\setminus\left(\cup_{t\in G}tHt^{-1}\right) = G\setminus\left(\cup_{r\in R}rHr^{-1}\right),
    \end{equation}
    and
    \begin{equation}
        \abs{N} = \abs{G}-\abs{R}(\abs{H}-1)-1 = g-\frac{g}{h}(h-1)-1 = \frac{g}{h}-1.
    \end{equation}
    \item[(b)] Note that the conjugacy class of $1$ is always $1$. Let's define $\tilde{f}$ by
    \begin{equation}
        \tilde{f}(g) = \begin{cases}
        f(h) & \textrm{if there exists $t\in G$ and $h\in H$ such that } tht^{-1} = g\\
        f(1) & \textrm{if $g\in N$}.
        \end{cases}
    \end{equation}
    I'll show that this is well-defined class function. If $s\not\in N$, there exists $t\in G$ and $h\in H$ such that $tht^{-1} = s$. Assume there exists another $t_1\in G$ and $h_1\in H$ making $t_1h_1t_1^{-1} = g$, then $h_1 = t_1^{-1}th(t_1^{-1}t)^{-1}$, so $f(h_1) = f(h)$. Therefore, it is well-defined function. Now, assume $h_1$ and $h_2$ are in a same conjugacy class in $G$, i.e. there exists $t\in G$ such that $th_1t^{-1} = h_2$. If $h_2\not\in N$, $h_1\not\in N$ and we get $\tilde{f}(h_1)=\tilde{f}(h_2)$ by the definition. If $h_2\in N$, then $h_1\in N$ and it also have same $\tilde{f}$ value. Therefore, it is a class function.
    
    I'll show the uniqueness. However, by the definition of $N$, there exists $t\in G$ such that $t^{-1}st\in H$ if and only if $s\in N^c$. Therefore, $\tilde{f}$ which extends $f$ and $\tilde{f}\equiv f(1)$ on $N$ is unique.
    \item[(c)] I'll divide the case into three cases: $s=1$, $s\in N\setminus\{1\}$, and $s\in N^c\setminus\{1\}$.
    \begin{enumerate}
        \item If $s=1$, $\tilde{f}(1) = \frac{g}{h}f(1) - f(1)\left(\frac{g}{h}-1\right) = f(1)$.
        \item If $s\in N\setminus\{1\}$, $\tilde{f}(s) = 0 - f(1)(0-1) = f(1)$, cf. the definition of induced character.
        \item If $s\in N^c\setminus\{1\}$, then there exists $t\in G$ and $h\in H$ such that $tht^{-1} = s$. Since $s\neq 1$ and $H$ is a Frobenius subgroup of $G$, the only possible $t'$ making $(t')^{-1}st'\in H$ is $t'\in tH$. Therefore,
        \begin{equation}
            \Ind_H^G f(s) = \frac{1}{\abs{H}}\sum_{\substack{t'\in G\\(t')^{-1}st'\in H}}f((t')^{-1}st') = f(t^{-1}st) = f(h),
        \end{equation}
        and
        \begin{equation}
            \tilde{f}(s) = f(h) - f(1)(1_H(h)-1_G(s)) = f(h).
        \end{equation}
    \end{enumerate}
    \item[(d)] In the proof of (c), note that for any $s\in H$, $\tilde{f}(s) = \Ind_H^Gf(s)$. Therefore, $\Res_H\Ind_H^Gf(s) = f(s)$ for any class function $f$. There is no confusion about the subscript $G$ and $H$ of inner product, so I'll omit it.
    \begin{equation}
    \begin{split}
        \langle \tilde{f}, \tilde{f}\rangle &= \langle \Ind_H^G f - f(1)(\Ind_H^G(1)-1), \Ind_H^G f - f(1)(\Ind_H^G(1)-1)\rangle \\
        &=\langle \Ind_H^G f, \Ind_H^G f \rangle - 2f(1)\langle \Ind_H^G f, \Ind_H^G(1)-1\rangle + \left(f(1)\right)^2\langle \Ind_H^G(1)-1, \Ind_H^G(1)-1\rangle\\
        &=\langle \Ind_H^G f, \Ind_H^G f \rangle - 2f(1)\langle \Ind_H^G f, \Ind_H^G(1)-1\rangle + \left(f(1)\right)^2\left(\langle \Ind_H^G(1), \Ind_H^G(1)\rangle-1\right)\\
        &=\langle f, \Res_H\Ind_H^G f \rangle - 2f(1)\left(\langle f, \Res_H\Ind_H^G(1)\rangle-\langle f, 1_H\rangle\right)\\
        &\phantom{=}+\left(f(1)\right)^2\left(\langle 1_H, \Res_H\Ind_H^G(1)\rangle - \langle 1_H, 1_H\rangle\right)\\
        &=\langle f, f\rangle.
    \end{split}
    \end{equation}
    In the third to fourth line, I used the Frobenius reciprocity formula.
    \item[(e)] Since $\tilde{f}(1)=f(1)$ and $f(1)\in\mathbb{Z}_{\geq 0}$, $\tilde{f}\in \mathbb{Z}_{\geq 0}$. Also, $\langle \tilde{f}, \tilde{f}\rangle = \langle f, f\rangle = 1$. Note that we don't know whether $\tilde{f}$ is a character. 
    
    For any irreducible representation $\varphi$ of $G$,
    \begin{equation}
    \begin{split}
        \langle \varphi, \tilde{f}\rangle &= \langle \varphi, \Ind_H^G f\rangle  - f(1)\langle \varphi, \Ind_H^G 1_H - 1_G\rangle \\
        &=\langle \Res_H\varphi,  f\rangle  - f(1)\left(\langle \Res_H\varphi, 1_H\rangle - \langle \varphi, 1_G\rangle\right).
    \end{split}
    \end{equation}
    Since $f$ and $1_H$ are irreducible characters of $H$, $\Res_H \varphi$ is a character, and $1_G$ is an irreducible character of $1_G$, all the terms in the above equation are in $\mathbb{Z}_{\geq 0}$. Therefore, $\tilde{f}$ is a linear combination with integer coefficients of irreducible character of $G$. Note that we don't know whether $\tilde{f}$ is a character in this step since the coefficient may be negative. Let's write $\tilde{f} = \sum_{i=1}^m n_i \varphi_i$ where $n_i\in\mathbb{Z}$ and $\varphi_i$ are irreducible characters of $G$, then
    \begin{equation}
        \langle \tilde{f}, \tilde{f}\rangle = \sum_{i=1}^m n_i^2
    \end{equation}
    and it is $1$ if and only if $n_i=0$ except one $i$, which should be $\pm 1$. Also, if the coefficient were $1$, $\tilde{f}(1)<0$, which is not true. Therefore, $\tilde{f}$ is an irreducible character of $G$.
    
    Since $\tilde{f}(1) = f(1) = \tilde{f}(n)$ for all $n\in N$ and $\tilde{f}(1)$ represents the degree of the representation corresponding to the character, using exercise 6.7, we get $\rho(s) = 1$ for all $s\in N$.
    
    \item[(f)] Let $\rho:H\rightarrow GL(V)$ be an irreducible representation of $H$. The character $\chi$ of $\rho$ extends to an irreducible character $\chi'$ of $G$; let the irreducible representation $\rho'$. Since $\Res_H \rho'$ have the same character as $\rho$ as checked above, it is isomorphic to $\rho$, which explains the term "extension".
    
    Any linear representation $\rho$ of $H$ is decomposed into irreducible ones and the character becomes the sum of the irreducible characters. Since $\Ind_H^G$ and $\Res_H$ are linear maps for class functions, the above arguments applies to each irreducible character consisting the character function. Finally, the sum of extended irreducible character is a character of some linear representation $\rho'$ of $G$, which can be considered as a extension of $\rho$ since $\Res_H \rho'$ is isomorphic to  $\rho$ by the same reason above.
    
    $\rho' = \oplus_{i=1}^m \rho'_i$ where $\rho'_i$ is the extension of irreducible representation $\rho_i$ of $H$ and $\rho'_i(n) = 1$ for all $n\in N$ and $i$, so $\rho'(n) = 1$. Therefore, $(\rho')^{-1}(1)$ contains $N\cup {1}$. If $N\cup {1}$ forms a subgroup in $G$, it is a normal subgroup of $G$ as a subgroup of the kernel. Also, $(N\cup \{1\})\cap H = 1$, so if $n_1h_1 = n_2h_2$ in $(N\cup \{1\})H$, $n_2^{-1}n_1 = h_2h_1^{-1}\in (N\cup \{1\})\cap H$, so $n_1=n_2$ and $h_1=h_2$, which implies $\abs{(N\cup \{1\})H} = \abs{N\cup \{1\}}\abs{H} = g$ and $(N\cup \{1\})H = G$ as a set. As a result, $G$ is the semidirect product of $H$ and $N\cup \{1\}$.
    
    To end the proof, I need to show that $N\cup \{1\}$ is a subgroup of $G$. Consider a regular representation $r_H$ of $H$. The character $\chi$ of $r_H$ is $0$ except at $1$, $\abs{H}$. Therefore,
    \begin{equation}
        \tilde{\chi}(s) = \begin{cases}
        n & s\in N\cup \{1\}\\
        0 & s\in (N\cup \{1\})^c.
        \end{cases}
    \end{equation}
    which means that the kernel of extension of $r_H$ is exactly $N\cup \{1\}$. It shows that $N\cup \{1\}$ is a normal subgroup of $G$.
    \item[(g)] I'll write $H^* = H\setminus\{1\}$ and $A^* = A\setminus \{1\}$. If $H$ is a Frobenius subgroup of $G$, then the latter statement is easily followed by the definition since $A^*\cap H^* = 1$, so $t\not\in H$ and $tst^{-1}\neq s$. Hence, I'll show the converse. Assume there exists $h\in H^*$ and $t\not\in H$ such that $tht^{-1}\in H$. Writing $G = A\rtimes H$, $t = (a',1)(1,h')$ for some $a'\in A^*$ and $h'\in H$. Therefore,
    \begin{equation}
        tht^{-1} = (a', 1)(1,h'h(h')^{-1})((a')^{-1}, 1)\in H
    \end{equation}
    Since $h\in H^*$, $h'h(h')^{-1}\in H^*$. It means that $a'((h'h(h')^{-1})\cdot (a')^{-1}) = 1$ in $A$. By replacing $t = ((a')^{-1},1)$, $s = (1,h'h(h')^{-1})$,
    \begin{equation}
        sts^{-1}(1,h'h(h')^{-1})((a')^{-1},1)(1,h'h^{-1}(h')^{-1}) = ((h'h(h')^{-1})\cdot (a')^{-1}, 1) = ((a')^{-1}, 1) = t
    \end{equation}
    which contradicts the assumption. Therefore, $H$ is a Frobenius subgroup of $G$.
\end{enumerate}

\noindent \textbf{9}(\textbf{S} 7.9)
    Let's first configure what is $H_s$. For $s = \begin{pmatrix}\alpha & \beta\\ \gamma & \delta
    \end{pmatrix}$ and $t = \begin{pmatrix}
    a & b\\ 0 & d
    \end{pmatrix}$, $sts^{-1}$ is
\begin{equation}
    \begin{pmatrix}
    \alpha & \beta\\
    \gamma & \delta
    \end{pmatrix}\begin{pmatrix}
    a & b \\ 0 & d
    \end{pmatrix}\begin{pmatrix}
    \delta & -\beta\\
    -\gamma & \alpha
    \end{pmatrix} = \begin{pmatrix}
    a\alpha\delta - b\alpha\gamma - d\gamma\beta & -a\alpha\beta + b\alpha^2+d\alpha\beta\\
    a\gamma\delta - b\gamma^2 - d\gamma\delta & -a\beta\gamma + b\alpha\gamma + d\alpha\delta
    \end{pmatrix}
\end{equation}
To make it in $H$, we need to impose $a\gamma\delta - b\gamma^2 - d\gamma\delta = 0$. Assume $s\not\in H$, then $\gamma\neq 0$, so $\delta(a-d)-b\gamma = 0$. Therefore,
\begin{equation}
    \begin{pmatrix}
    a\alpha\delta - b\alpha\gamma - d\gamma\beta & -a\alpha\beta + b\alpha^2+d\alpha\beta\\
    a\gamma\delta - b\gamma^2 - d\gamma\delta & -a\beta\gamma + b\alpha\gamma + d\alpha\delta
    \end{pmatrix} = \begin{pmatrix}
    d & -a\alpha\beta + b\alpha^2+d\alpha\beta\\
    0 & a
    \end{pmatrix}
\end{equation}
using $\alpha\delta-\gamma\beta = 1$. Also,
\begin{equation}
    \gamma(-a\alpha\beta + b\alpha^2+d\alpha\beta) = \alpha(-a\beta\gamma + \delta(a-d)\alpha + d\gamma\beta) = \alpha(a-d),
\end{equation}
so
\begin{equation}
    \begin{pmatrix}
    d & -a\alpha\beta + b\alpha^2+d\alpha\beta\\
    0 & a
    \end{pmatrix} = \begin{pmatrix}
    d & \alpha\gamma^{-1}(a-d)\\
    0 & a
    \end{pmatrix}
\end{equation}
Therefore, 
\begin{equation}
    H_s = \left\{\begin{pmatrix}
    d & \alpha\gamma^{-1}(a-d) \\ 0 & a
    \end{pmatrix}:ad=1\right\}.
\end{equation}
Therefore, for $s\not\in H$,
\begin{equation}
\begin{split}
    \langle \rho^s, \Res_{H_s}(\rho) \rangle &= \frac{1}{\abs{H_s}}\sum_{t\in H_s}\rho^s(t^{-1})\left(\Res_{H_s}\rho\right)(t) \\
    &=\frac{1}{k}\sum_{a\in k\setminus\{0\}}\chi_\omega^s\left(\begin{pmatrix}
    a & -\alpha\gamma^{-1}(a-d)\\ 0 & d
    \end{pmatrix}\right)\chi_\omega\left(\begin{pmatrix}
    d & \alpha\gamma^{-1}(a-d) \\ 0 & a
    \end{pmatrix}\right)\\
    &=\frac{1}{k}\sum_{a\in k\setminus\{0\}}\chi_\omega\left(\begin{pmatrix}
    d & -b\\ 0 & a
    \end{pmatrix}\right)\chi_\omega\left(\begin{pmatrix}
    d & \alpha\gamma^{-1}(a-d) \\ 0 & a
    \end{pmatrix}\right) = \frac{1}{\abs{k}}\sum_{d\in k\setminus\{0\}}\omega^2(d)
\end{split} 
\end{equation}
Assume $\omega^2\neq 1$. Since $k$ is finite field, $k^*$ is a cyclic group about an element $x\in k$. (cf. \textbf{D\&F} proposition 9.18.) Therefore, $\omega(x)$ is a root of unity such that the order of $\omega(x)$ divides $\abs{k}$. It shows that
\begin{equation}
    \sum_{d\in k\setminus\{0\}}\omega^2(d) = \sum_{i=1}^{\abs{k}}\omega^2(x^i) = 0.
\end{equation}
Now, the induced representation satisfies all the conditions in proposition 23, so it is irreducible. (Since $\chi_\omega$ is the character of degree 1, the condition (a) is automatically satisfied.)
%________________________________________________________________________
\end{document}

%================================================================================