%Calculus Homework
\documentclass[a4paper, 12pt]{article}

%================================================================================
%Package
	\usepackage{amsmath, amsthm, amssymb, latexsym, mathtools, mathrsfs, physics, amsfonts}
	\usepackage{dsfont, txfonts, soul, stackrel, tikz-cd, graphicx, titlesec, etoolbox}
	\DeclareGraphicsExtensions{.pdf,.png,.jpg}
	\usepackage{fancyhdr}
	\usepackage[shortlabels]{enumitem}
	\usepackage[pdfmenubar=true, pdfborder	={0 0 0 [3 3]}]{hyperref}
	\usepackage{kotex}

%================================================================================
\usepackage{verbatim}
\usepackage{physics}
\usepackage{makebox}
\usepackage{pst-node}

%================================================================================
%Layout
	%Page layout
	\addtolength{\hoffset}{-50pt}
	\addtolength{\headheight}{+10pt}
	\addtolength{\textwidth}{+75pt}
	\addtolength{\voffset}{-50pt}
	\addtolength{\textheight}{+75pt}
	\newcommand{\Space}{1em}
	\newcommand{\Vspace}{\vspace{\Space}}
	\newcommand{\ran}{\textrm{ran }}
	\setenumerate{listparindent=\parindent}

%================================================================================
%Statement
	\newtheoremstyle{Mytheorem}%
	{1em}{1em}%
	{\slshape}{}%
	{\bfseries}{.}%
	{ }{}

	\newtheoremstyle{Mydefinition}%
	{1em}{1em}%
	{}{}%
	{\bfseries}{.}%
	{ }{}

	\theoremstyle{Mydefinition}
	\newtheorem{statement}{Statement}
	\newtheorem{definition}[statement]{Definition}
	\newtheorem{definitions}[statement]{Definitions}
	\newtheorem{remark}[statement]{Remark}
	\newtheorem{remarks}[statement]{Remarks}
	\newtheorem{example}[statement]{Example}
	\newtheorem{examples}[statement]{Examples}
	\newtheorem{question}[statement]{Question}
	\newtheorem{questions}[statement]{Questions}
	\newtheorem{problem}[statement]{Problem}
	\newtheorem{exercise}{Exercise}[section]
	\newtheorem*{comment*}{Comment}
	%\newtheorem{exercise}{Exercise}[subsection]

	\theoremstyle{Mytheorem}
	\newtheorem{theorem}[statement]{Theorem}
	\newtheorem{corollary}[statement]{Corollary}
	\newtheorem{corollaries}[statement]{Corollaries}
	\newtheorem{proposition}[statement]{Proposition}
	\newtheorem{lemma}[statement]{Lemma}
	\newtheorem{claim}{Claim}
	\newtheorem{claimproof}{Proof of claim}[claim]
	\newenvironment{myproof1}[1][\proofname]{%
  \proof[\textit Proof of problem #1]%
}{\endproof}

%================================================================================
%Header & footer
	\fancypagestyle{myfency}{%Plain
	\fancyhf{}
	\fancyhead[L]{}
	\fancyhead[C]{}
	\fancyhead[R]{}
	\fancyfoot[L]{}
	\fancyfoot[C]{}
	\fancyfoot[R]{\thepage}
	\renewcommand{\headrulewidth}{0.4pt}
	\renewcommand{\footrulewidth}{0pt}}

	\fancypagestyle{myfirstpage}{%Firstpage
	\fancyhf{}
	\fancyhead[L]{}
	\fancyhead[C]{}
	\fancyhead[R]{}
	\fancyfoot[L]{}
	\fancyfoot[C]{}
	\fancyfoot[R]{\thepage}
	\renewcommand{\headrulewidth}{0pt}
	\renewcommand{\footrulewidth}{0pt}}

	\pagestyle{myfency}

%================================================================================

%***************************
%*** Additional Command ****
%***************************

\DeclareMathOperator{\cl}{cl}
\DeclareMathOperator{\sgn}{sgn}
\DeclareMathOperator{\co}{co}
\DeclareMathOperator{\ball}{ball}
\DeclareMathOperator{\wk}{wk}
\DeclareMathOperator{\spn}{span}
\DeclarePairedDelimiter{\ceil}{\lceil}{\rceil}
\DeclarePairedDelimiter\floor{\lfloor}{\rfloor}
\newcommand{\quotZ}[1]{\ensuremath{\mathbb{Z}/p^{#1}\mathbb{Z}}}
%================================================================================
%Document
\begin{document}
\thispagestyle{myfirstpage}
\begin{center}
	\Large{HW\#4}
\end{center}
박성빈, 수학과, 20202120

Notation: For each problem, I'll follow the notations in the problem if there is no additional mention.

\noindent \textbf{1}(\textbf{D\&F} 18.3.11)
Let $\phi$ be the irreducible representation of $G$ on $V$ generating $\chi$. Let $z$ is in the center of $G$, then we get
\begin{equation}
    \phi(z)\phi(g) = \phi(g)\phi(z)
\end{equation}
for all $g\in G$, which means that $\phi(z)$ is an $\mathbb{C}G$ module automorphism on $V$. By schur's lemma, it shows that $\phi(z)$ is a homothety, i.e., for $\lambda\in \mathbb{C}$, $\phi(z) = \lambda\cdot\mathrm{id}$, so $\chi(z) = \lambda\chi(1)$. Since $\abs{z}<\infty$, $\lambda^{\abs{z}}=1$ and $\lambda$ is some root of unity in $\mathbb{C}$.\\

\noindent \textbf{2}(\textbf{D\&F} 18.3.15)
This is basis-dependent argument(private communication with TA): For a cyclic group $G=<\sigma:\sigma^3 = 1>$, consider a group representation $\phi$ on $F^2$ given by
\begin{equation}
    \rho_\sigma = \begin{pmatrix}
    1 & 0\\
    0 & \exp\left(\frac{2\pi i}{3}\right)
    \end{pmatrix}.
\end{equation}
I'll denote $u = \exp\left(\frac{2\pi i}{3}\right)$. For $A = \begin{pmatrix}a & b\\ c & d\end{pmatrix}\in GL_2F$, the similar transformation is given by
\begin{equation}
    A^{-1}\rho_\sigma A = \frac{1}{ad-bc}\begin{pmatrix}
    ad-ubc & bd(1-u)\\
    ac(-1+u) & uad-bc
    \end{pmatrix}.
\end{equation}
For arbitrary nonzero $f\in F$, if I set $a=1+f$, $b=c=d=1$, then
\begin{equation}
    \frac{ad-ubc}{ad-bc} = \frac{f-u + 1}{f} = \frac{-u+1}{f} + 1.
\end{equation}
If $\mathbb{Q}(\varphi)$ implies collecting all entries of $\varphi$ upto similar transformation, it means that $F$ is finite extension of $\mathbb{Q}[u]$, which is contradiction. Therefore, I'll fix a basis for $GL_n(F)$.

Define
\begin{equation}
    A = \cup_{s\in G}\cup_{a^s_{ij}\textrm{ is an entry of }\varphi(s)}\{a^s_{ij}\},
\end{equation}
then it is finite since $\abs{G}<\infty$ and $\varphi(s)\in GL_n(F)$. Furthermore, $a^s_{ij}\in F$, so each $[\mathbb{Q}(a^s_{ij}):\mathbb{Q}]<\infty$. Therefore, $[\mathbb{Q}(A):\mathbb{Q}]<\infty$, which shows that $\mathbb{Q}(\varphi)$ is finite extension of $\mathbb{Q}$.\\

\noindent \textbf{3}(\textbf{D\&F} 18.3.16)
Fix $s\in G$. Since $\sigma$ is a automorphism on $F$, for $a_i,b_i\in F$,
\begin{equation}
    \sigma\left(\sum_i a_ib_i\right) = \sum_i \sigma(a_i)\sigma(b_i).
\end{equation}
Therefore, $\sigma(AB)=\sigma(A)\sigma(B)$ for $A,B\in GL_n(F)$, which shows that $\varphi^\sigma$ is a group homomorphism from $G$ to $GL_n(F)$, so $\varphi^\sigma$ is a representation if $\varphi$ is a representation. Furthermore, $\tr(\sigma(A)) = \sigma(\tr(A))$ by the same reason, so we get the character of $\varphi^\sigma = \sigma\circ \psi$.

\noindent \textbf{4}(\textbf{D\&F} 18.3.17)
Since $(\varphi^\sigma)^{\sigma^{-1}} = \varphi$, it is enough to show that $\varphi$ is irreducible implies $\varphi^\sigma$ is irreducible.

Assume $\varphi^\sigma$ is not irreducible, so there exists a proper subspace $W$ of $V$ such that $\varphi^\sigma|_W$ is an automorphism on $W$. It means that there exists a complement $W^0$ of $W$ which also satisfies $\varphi^\sigma|_{W^0}\in Aut(W^0)$. Therefore, the matrix form of $\varphi$ can be decomposed into smaller block matrices, which have determinant non-zero. Taking $\sigma^{-1}$ to each entries of the decomposed matrix, we get the same block decomposed matrices with non-zero determinant since $\sigma$ is field isomorphism. It shows that $\varphi$ is not irreducible, which is contradiction. It ends the proof.\\

\noindent \textbf{5}(\textbf{D\&F} 19.3.1)
For basis $1\otimes e_1, 1\otimes e_2, 1\otimes e_3, (1~2)\otimes e_1, (1~2)\otimes e_2, (1~2)\otimes e_3\}$, the matrix representation is given by the following: for
\begin{equation}
    \begin{split}
        P_1 &= \begin{pmatrix}
        0 & 0 & 1\\
        1 & 0 & 0\\
        0 & 1 & 0
        \end{pmatrix}\\
        P_2 &= \begin{pmatrix}
        0 & 1 & 0\\
        0 & 0 & 1\\
        1 & 0 & 0
        \end{pmatrix},
    \end{split}
\end{equation}
we get
\begin{equation}
    \begin{split}
        1&\mapsto \begin{pmatrix}
            I_3 & 0\\
            0 & I_3
        \end{pmatrix},\\
        (1~2)&\mapsto\begin{pmatrix}
            0 & I_3\\
            I_3 & 0
        \end{pmatrix},\\
        (1~3)&\mapsto\begin{pmatrix}
            0 & \varphi((1~2~3))\\
            \varphi((1~3~2)) & 0
        \end{pmatrix} = \begin{pmatrix}
            0 & P_1\\
            P_2 & 0
        \end{pmatrix},\\
        (2~3)&\mapsto\begin{pmatrix}
            0 & \varphi((1~3~2))\\
            \varphi((1~2~3)) & 0
        \end{pmatrix} = \begin{pmatrix}
            0 & P_2\\
            P_1 & 0
        \end{pmatrix},\\
        (1~2~3)&\mapsto\begin{pmatrix}
            \varphi((1~2~3)) & 0\\
            0 & \varphi((1~3~2))
        \end{pmatrix} = \begin{pmatrix}
            P_1 & 0\\
            0 & P_2
        \end{pmatrix},\textrm{ and}\\
        (1~3~2)&\mapsto\begin{pmatrix}
            \varphi((1~3~2)) & 0\\
            0 & \varphi((1~2~3))
        \end{pmatrix} = \begin{pmatrix}
            P_2 & 0\\
            0 & P_1
        \end{pmatrix}.\\
    \end{split}
\end{equation}

\noindent \textbf{6}(\textbf{D\&F} 19.3.2(a))
Since induced representation is unique upto isomorphism, it is enough to calculate the character with some fixed representatives of $G/H$. Since $<(1~2)>$ has order $2$, it has two irreducible representations: trivial and non-trivial one. The non-trivial irreducible character $\psi$ is $\psi(1)=1$ and $\psi((1~2)) = -1$. For representatives $\{1, (1~3),(2~3)\}$, the induced character $\Psi$ is
\begin{equation}
    \begin{split}
        \Psi(1) &= 3\psi(1) = 3\\
        \Psi((1~2)) &= \Psi((1~3)) = \Psi((2~3)) = \psi((1~2)) = -1\\
        \Psi((1~2~3)) &= \Psi((1~3~2)) =0.
    \end{split}
\end{equation}
Using the character table of $S_3$ in section 19.1, we get $\Psi = \chi_2+\chi_3$.\\

\noindent \textbf{7}(\textbf{D\&F} 19.3.4)
Let $R = \{1, g_1, \ldots, g_m\}$ be the representation set of $G/H$ and $\Phi$ be the induced representation of $\varphi$ which is a representation on $V$. Fix $g_i\in R$. For $v\in V$ and $n\in N\leq H$, there exists $n'\in N$ such that $ng_i = g_in'$ since $N$ is a normal subgroup of $G$. Therefore,
\begin{equation}
    n\cdot(g_i\otimes v) = ng_i\otimes v = g_in'\otimes v = g_i\otimes (n'\cdot v) =g_i\otimes v.
\end{equation}
It shows that $N$ is contained in the kernel of induced presentation.\\

\noindent \textbf{8}(\textbf{S} 3.3.4)
From example 1, we know that the regular representation $\rho$ on $G$ is induced by the regular representation $\theta$ on $H$. Let's decompose $\rho=\oplus_{i=1}^n \rho_i$ and $\theta = \oplus_{i=1}^m \theta_i$ into irreducible representations. By Corollary 5.1 in chapter 2, every irreducible representations on $G$ and $H$ is contained in $\rho$ and $\theta$.

Let $\rho'_i$ be the induced representation of $\theta_i$, then by example 3, we know that $\oplus_{i=1}^m \rho'_i$ is induced by $\oplus_{i=1}^m \theta_i$. By the uniqueness of induced representation, $\oplus_{i=1}^m \rho'_i$ and $\rho$ are isomorphic, so each irreducible components of $\rho$ is contained in some $\rho'_i$, which shows the statement in the problem.

Since $A\leq G$ is abelian, each $\theta_i$ has degree 1. It means that the induced representation $\rho'_i$ of $\theta_i$ has degree $g/a$ for all $i$, so each $\rho_i$ should have degree not greater than $g/a$.\\

\noindent \textbf{9}(\textbf{S} 3.3.5)
Before start, I'll show that $\rho$ is a well-defined representation. For $s_1,s_2\in G$, $(\rho_{s_1}(\rho_{s^{-1}_1})f)(u) = f(us_1s_1^{-1}) = f(u)$ and by the same reason, $(\rho_{s^{-1}_1}(\rho_{s_1})f)(u) = f(u)$, so $\rho_s\in GL(V)$ for all $s\in G$. Also,
\begin{equation}
    (\rho_{s_1s^{-1}_2}f)(u) = f(us_1s^{-1}_2) = (\rho_{s_1}(\rho_{s^{-1}_2}f))(u) = (\rho_{s_1}(\rho^{-1}_{s_2}f))(u).
\end{equation}
Therefore, it is a well-defined group action, i.e. group homomorphism from $G$ to $GL(V)$.

To show $w\mapsto f_w$ is an isomorphism, I'll first show that $f_w\in V$. For $t\in H$ and $u\in G$, if $tu\in H$, which means that $u\in H$,
\begin{equation}
    f_w(tu) = \theta_{tu}w = \theta_t\theta_uw = \theta_tf_w(u).
\end{equation}
If $tu\not\in H$, $f_w(u) = 0$ since $u\not\in H$, so it also satisfies $f_w(tu) = \theta_tf_w(u)$. Therefore, $f_w\in V$.

To show the isomorphism from $W$ to $W_0$, it is enough to show that $\varphi:w\mapsto f_w$ is injection from $W$. If $w_1\neq w_2$, then $f_{w_1}(1)=w_1\neq w_2 = f_{w_2}(1)$, so $f_{w_1}\neq f_{w_2}$.

Now, I'll show that $\rho$ is induced by $\theta$. Let's first fix a representatives $R=\{1=\sigma_1, \ldots, \sigma_n\}\in G$ of $G/H$. I'll first show that $\{\rho_{\sigma_i}f_{w_j}\}$ forms a basis of $V$. For linearly independence, assume there exists $a_{ij}\in\mathbb{C}$ satisfying
\begin{equation}
    \sum_{i,j}a_{ij}\rho_{\sigma_i} f_{w_j} = 0.
\end{equation}
It means that for $u\in G$,
\begin{equation}
    \sum_{i,j}a_{ij}\rho_{\sigma_i} f_{w_j}(u) = \sum_{i,j}a_{ij} f_{w_j}(u\sigma_i) = \sum_j\sum_{u\sigma_i\in H}a_{ij} \theta_{u\sigma_i}w_j = 0.
\end{equation}
Note that $u$ acts on $G/H$ by permutation, so there exists only one $i_0$ such that $u\sigma_{i_0}\in H$, which means that
\begin{equation}
    \sum_ja_{i_0j} \theta_{u\sigma_{i_0}}w_j = 0.
\end{equation}
Since $\theta_{u\sigma_{i_0}}$ is automorphism on $W$ and $w_j$ is a basis on $W$, $a_{i_0j} = 0$ for all $j$. Since this is true for arbitrary $u\in G$ and $G$ acts on $G/H$ transitively, we get $a_{ij}=0$ for all $i,j$. Therefore, $\{\rho_{\sigma_i}f_{w_j}\}$ is linearly independent.

Now, I'll show that the $\{\rho_{\sigma_i}f_{w_j}\}$ spans $V$. Fix arbitrary $f\in V$. Even though I chose left coset of $G/H$, it also works as a right coset of $H\backslash G$. Therefore, it is enough to show that I can generate $f(\sigma_i)$ using the basis to generate the $f$: for any $u\in G$, there exists $\sigma\in R$ and $t\in H$ such that $tu=\sigma$, so $f(u) = f(t^{-1}tu) = \theta_{t^{-1}}f(\sigma)$. Let's write $f(\sigma_i) = \sum_{i,j}a_{ij}w_j$. If I set
\begin{equation}
    \phi = \sum_{i,j}a_{ij}\rho_{\sigma^{-1}_i}f_{w_j},
\end{equation}
for any $\sigma_k\in R$,
\begin{equation}
    \phi(\sigma_k) = \sum_{i,j}a_{ij}\rho_{\sigma^{-1}_i}f_{w_j}(\sigma_k) = \sum_{i,j}a_{ij}f_{w_j}(\sigma_k\sigma^{-1}_i) = \sum_j a_{kj}w_j = f(\sigma_k).
\end{equation}
Therefore, $\phi=f$. 

By writing $W_{\sigma_i} = \mathrm{span}\{\rho_{\sigma_i}f_{w_1}, \ldots, \rho_{\sigma_i}f_{w_m}\}$, we get $V = \oplus_{\sigma\in R}W_\sigma$, which means that $\rho$ is induced by $\theta$.\\

\noindent \textbf{10}(\textbf{S} 3.3.6)
Since $G=H\times K$, $hk=kh$ for all $h\in H$ and $k\in K$. It shows that $H \trianglelefteq G$ and $G/H\simeq K$. Therefore, We can take the representatives of $G/H$ by the elements of $K$, and for any $k\in K$ and $h\in H$, we get $h(kH) = kH$. I'll write the representatives $R = \{k_1, \ldots, k_n\}$ with $n=\abs{K}$ if enumerating is necessary.

Let $W$ and $L$ be vector spaces such that $\theta:H\rightarrow GL(W)$ and $r_K:K\rightarrow GL(L)$, i.e. $L$ has basis $\{e_k\}_{k\in K}$. Since $\rho:G\rightarrow GL(V)$ is induced by $\theta$, we can write
\begin{equation}
    V= \oplus_{k\in K}W_k.
\end{equation}
Let's construct a vector space isomorphism $\varphi:W\otimes L\rightarrow V$ by
\begin{equation}
    \varphi(w\otimes l) = \varphi(\sum_{k\in K}a_k(w\otimes e_k)) = \sum_{k\in K}a_k\rho_k w.
\end{equation}
Indeed, we know that $W\otimes L$ is a vector space having basis as a simple tensor of basis elements in $W$ and $L$, this is well-defined map, and this is surjective since for any $\sum_{k\in K}w_k\in V$ such that $w_k\in W_k$,
\begin{equation}
    \varphi:\sum_{k\in K}\rho_{k^{-1}}w_k\otimes e_k\mapsto \sum_{k\in K}w_k.
\end{equation}
By dimension analysis, $\varphi$ is a vector space isomorphism. Finally, this is isomorpshim of representation between $\theta\otimes r_K$ and $\rho$: with the same notation above and $s = hk_i\in G$ for some $k_i\in K$ and $h\in H$,
\begin{equation}
    \begin{split}
        \rho_s(\varphi(w\otimes l)) &=\sum_{k\in K}a_k\rho_s\rho_kw = \sum_{k\in K} a_k\rho_{k_ik}\rho_hw\\
        \varphi\left(\left(\theta\otimes r_K\right)_s(w\otimes l)\right) &= \varphi\left(\theta_h(w)\otimes (r_K)_{k_i}\left(\sum_{k\in K}a_ke_k\right)\right) = \varphi\left(\theta_h(w)\otimes \left(\sum_{k\in K}a_ke_{k_ik}\right)\right) = \sum_{k\in K}a_k\rho_{k_ik}\rho_hw.
    \end{split}
\end{equation}
(To write it more precise, I need to introduce the inclusion map $i:W\rightarrow V$ and use $i(\theta_h(w)) = \rho_h(i(w))$.) Therefore, $\theta\otimes r_K$ and $\rho$ are isomorphic.
%________________________________________________________________________
\end{document}

%================================================================================